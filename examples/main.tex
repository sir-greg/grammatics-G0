\documentclass[12pt]{article}
\usepackage[T1, T2A]{fontenc}
\usepackage[utf8]{inputenc}
\usepackage[russian]{babel}
\usepackage{amsmath}
\usepackage{amsthm}
\usepackage{amssymb}
\usepackage{esvect}
\usepackage{listings}
\usepackage{xcolor}
\usepackage{mathrsfs}
\usepackage{braket}

% for large comments
\usepackage{blindtext, xcolor}
\usepackage{comment}

% for inkscape pictures
\usepackage{import}
\usepackage{pdfpages}
\usepackage{transparent}
\usepackage{xcolor}

\newcommand{\incfig}[2][1]{%
    \def\svgwidth{#1\columnwidth}
    \import{./figures/}{#2.pdf_tex}
}

\pdfsuppresswarningpagegroup=1

% for systems of equations
\newenvironment{system_and}%
{\left\lbrace\begin{array}{@{}l@{}}}%
{\end{array}\right.}
% for unions of equations
\newenvironment{system_or}%
{\left\lbrack\begin{array}{@{}l@{}}}%
{\end{array}\right.}

\renewcommand{\C}{\mathbb{C}}
\newcommand{\R}{\mathbb{R}}
\newcommand{\Q}{\mathbb{Q}}
\newcommand{\Z}{\mathbb{Z}}
\newcommand{\N}{\mathbb{N}}

\newcommand{\floor}[1]{\left\lfloor #1 \right\rfloor}
\newcommand{\ceil}[1]{\left\lceil #1 \right\rceil}

% style of code listings
%\definecolor{codegreen}{rgb}{0,0.6,0}
%\definecolor{codegray}{rgb}{0.5,0.5,0.5}
%\definecolor{codepurple}{rgb}{0.58,0,0.82}
%\definecolor{backcolour}{rgb}{0.95,0.95,0.92}
%
%\lstdefinestyle{mystyle}{
%    backgroundcolor=\color{backcolour},
%    commentstyle=\color{codegreen},
%    keywordstyle=\color{magenta},
%    numberstyle=\tiny\color{codegray},
%    stringstyle=\color{codepurple},
%    basicstyle=\ttfamily,
%    breakatwhitespace=false,
%    breaklines=true,
%    captionpos=b,
%    keepspaces=true,
%    numbers=left,
%    numbersep=5pt,
%    showspaces=false,
%    showstringspaces=false,
%    showtabs=false,
%    tabsize=4
%}

\newtheorem{theorem}{\underline{Теорема}}[section]
\newtheorem{lemma}[theorem]{\underline{Лемма}}
\newtheorem{statement}{\underline{Утверждение}}[section]
\newtheorem{axiom}{\underline{Аксиома}}[section]
\newtheorem*{note}{\underline{Замечание}}
\newtheorem*{symb}{\underline{Обозначение}}
\newtheorem*{example}{\underline{Пример}}
\newtheorem*{consequence}{\underline{Следствие}}
\newtheorem*{solution}{\underline{Решение}}

\theoremstyle{definition}
\newtheorem{definition}{\underline{Определение}}[section]

\theoremstyle{definition}
\newtheorem{task}{\underline{Задача}}[section]



\begin{document}
    \newpage
    Будем считать, что грамматика $G_0 = <T, N, P, S>$ вычисляет $X$ и получает $Y$ $\iff$ из строки $SX!$ можно получить $Y$ (т. е. $'!'$ конечный символ), применяя правила $\in P$. По умолчанию будем считать, что $'!' \in T$ и $S \in N$. Кроме того, если в ходе описания правил $G$ появляется новый символ, будем считать его \underline{нетерминальным}.
    % Sources:
    \section{Сложение унарных чисел}
\[
    \underbrace{1\ldots 1}_{a}+\underbrace{1\ldots 1}_{b} \rightarrow \underbrace{1 \ldots 1}_{a + b}
\]
Грамматика $G = <T, N, P, S>$:
\begin{itemize}
    \item $T = \set{1, +}$
    \item $N = \set{S}$
    \item $P\colon$
        \begin{itemize}
            \item $S' 1\mapsto 1 S'$ --- движемся к символу $+$
            \item $S' + \mapsto \varepsilon$ --- удаляем $+$ и завершаем программу
        \end{itemize}
\end{itemize}


    \section{Удвоитель}
\[
    1^{n} \rightarrow 1^{2n}
\]
Тогда $G = <T, N, P, S>$ имеет вид:
\begin{itemize}
    \item $T = \set{1}$
    \item $N = \set{S, S'}$
    \item $P\colon$
        \begin{itemize}
            \item $S1 \mapsto 11S$ --- удваиваем единицы
            \item $S! \mapsto \varepsilon$ --- завершаем работу.
        \end{itemize}
\end{itemize}

    \section{Перевод из унарной с. и. в двоичную}
Грамматика $G = <T, N, P, S>$:
\begin{itemize}
    \item $T = \set{0, 1}$
    \item $N = \set{S}$
    \item $P\colon$
        \begin{itemize}
            \item $S \mapsto B0B'$ \\
                Если больше нет ещё не добавленных единиц.
                \begin{itemize}
                    \item $B'! \mapsto C_{clean_{left}}!$
                    \item $0 C_{clean_{left}} \mapsto C_{clean_{left}} 0$ 
                    \item $1 C_{clean_{left}} \mapsto C_{clean_{left}} 1$ 
                    \item $B C_{clean_{left}} \mapsto C_{clean_{right}}$
                    \item $C_{clean_{right}}0 \mapsto 0C_{clean_{right}}$
                    \item $C_{clean_{right}}1 \mapsto 1C_{clean_{right}}$
                    \item $C_{clean_{right}}! \mapsto \varepsilon$
                \end{itemize}
                Иначе:
                \begin{itemize}
                    \item $B'1 \mapsto 1' B_{x}B_{y}$
                    \item $11' \rightarrow 1'0$
                    \item $01' \mapsto 10$
                    \item $B1' \mapsto B 1$
                    \item $1 B_{x} \mapsto B_{x} 1$
                    \item $0 B_{x} \mapsto B_{x} 0$
                    \item $BB_{x} \mapsto B B_{z}$
                    \item $B_{z} 1 \mapsto 1 B_{z}$
                    \item $B_{z} 0 \mapsto 0 B_{z}$
                    \item $B_{z} B_{y} \mapsto B'$
                \end{itemize}
        \end{itemize}
        По итогу получаем двоичную запись числа.
\end{itemize}

    \section{Поиск подстроки в строке}
Даны строки $s$ и $t$. Необходимо найти первое вхождение строки $s$ в строку $t$ как подстроки.
\[
    s\#t \rightarrow \text{позиция первого вхождения $s$ в $t$}
\]
Если такой позиции не существует, грамматика оставит только $0$. 
Пусть $\Sigma$ --- алфавит строк $s$ и $t$. Поставим каждому символу из $\Sigma$ нетерминальный символ, и сформируем таким образом дополнительное множество $\Sigma'$. \\

Грамматика $G$ имеет вид:
\begin{itemize}
    \item $T = \Sigma \sqcup \set{\#, 1, 0}$
    \item $N = \set{S} \sqcup \Sigma'$
    \item $P\colon$ \\
        \begin{itemize}
            \item $S \mapsto S'C_{reset}$ 
            \item $\forall \sigma \in \Sigma\colon C_{reset}\sigma \rightarrow \sigma'C_{reset}$, где $\sigma$ и $\sigma'$ соответствующие друг другу символы (по построению $\Sigma$ и $\Sigma'$) \\
                Здесь отмечаем символы $s$, которые ещё не успели проверить на рав-во.
            \item $C_{passed}\# \mapsto C_{passed_s}\#$
            \item $C_{passed_s}\# \mapsto \# C_{passed_s}$ \\
                Здесь запоминаем, что дальше идут символы $t$.
            \item $C_{passed_s}\sigma' \mapsto \sigma'C_{passed_s}, \sigma' \in \Sigma'$ \\
                Пропускаем проверенные символы $t$
            \item $C_{passed_s}! \mapsto C_{kill}!$ \\
                Если $C_{passed_s}$ встретил $!$, то размер $\left|s\right| > \left|t\right|$, а следовательно, ответа нет. (действия $C_{kill}$ описываются далее).
            \item $C_{passed_s}\sigma \mapsto C_{= \sigma} \sigma', \sigma \in \Sigma \land \sigma' \in \Sigma'$ \\
                Запоминаем, с каким символом нужно сравнивать.
            \item $\alpha' C_{= \sigma} \mapsto C_{= \sigma} \alpha', \alpha' \in \Sigma' \lor \alpha' = \#$ \\
                Пропускаем все проверенные символы, символы
            \item $ C_{= \sigma}\beta' \mapsto C_{failed_{right}}\beta' , \beta \neq \sigma, \beta' \in \Sigma'$ \\
                Если проверка не пройдена, то делаем перезапуск, и превращаем первый символ $t$ в $\#$.
            \begin{itemize}
                \item $C_{failed_{right_s}} \beta' \mapsto \beta C_{failed_{right_s}}, \beta \in \Sigma$
                \item $C_{failed_{right_s}} \# \mapsto C_{failed_{\#}} \#$
                \item $C_{failed_{\#}} \# \mapsto \# C_{failed_{\#}}$
                \item $C_{failed_{\#}} a \mapsto \# C_{failed_{right_t}}$, \underline{здесь} $a \in \Sigma \cup \Sigma'$ \\
                    Если на этом моменте оказывается, что после $C_{failed_{right_t}}$ следует $!$, то ответа нет, а значит, мы должны стереть всё:
                \item $C_{failed_{right_t}}! \rightarrow C_{kill}!$
                \item $\alpha C_{kill} \rightarrow C_{kill}, \forall \alpha \in N \cup T \backslash \set{S'}$
                \item $S' C_{kill} ! \rightarrow 0$ \\
                    Иначе, готовимся к следующей итерации.
                \item $C_{failed_{right_t}}\beta' \mapsto \beta C_{failed_{right_t}}, \beta \in \Sigma, \beta' \in \Sigma'$
                \item $C_{failed_{right_t}}\beta \mapsto C_{left}\beta, \beta \in \Sigma$
                \item $\beta C_{left} \mapsto C_{left} \beta$, $\beta \neq S', \beta \in \Sigma \cup \set{\#}$
                \item $S'C_{left} \mapsto S'C_{reset}$
            \end{itemize} 
            \item $C_{=\sigma}\sigma' \mapsto \sigma C_{passed}, \sigma \in \Sigma, \sigma' \in \Sigma'$ \\
                Если проверка пройдена, то переходим к следующему символу.
            \item $C_{passed}\# \mapsto C_{output_{right}}$ \\
                Если последний символ $s$ прошёл проверку, то стираем всё, кроме $\#$ и заменяем $\#$ на $1$.
                \begin{itemize}
            \item $C_{output_{right}}\alpha \mapsto \alpha C_{output_{right}}, \forall \alpha \in T \cup N \backslash \set{!}$
            \item $C_{output_{right}}! \mapsto C_{output_{left}}!$
            \item $\alpha C_{output_{left}} \rightarrow C_{output_{left}}, \forall \alpha \neq \#$
            \item $\# C_{output_{left}} \mapsto 1$
            \item $\alpha 1 \mapsto 1, \alpha \neq \#$
            \item $ \# 1 \mapsto 1 1$
                \end{itemize}
        \end{itemize}
        По итогу этой части остаётся строка вида (если вхождение было найдено): 
        \[
        S' 1\ldots 1 !
        \]
        Где ответ представлен в унарной системе счисления. \\
        Перевод унарной записи в двоичную описан в предыдущей задаче.
\end{itemize}


\end{document}
